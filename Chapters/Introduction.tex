%%%%%%%%%%%%%%%%%%%%%%%%%%%%%%%%%%%%%%%%%
% Author: Terence L. van Zyl
%
% Affiliation: Wits Institute of Data Science 
%
% Template license:
% CC BY-NC-SA 3.0 (http://creativecommons.org/licenses/by-nc-sa/3.0/)
%
%%%%%%%%%%%%%%%%%%%%%%%%%%%%%%%%%%%%%%%%%

\chapter{Introduction} %Introduction

%----------------------------------------------------------------------------------------

% Define some commands to keep the formatting separated from the content 
\newcommand{\keyword}[1]{\textbf{#1}}
\newcommand{\tabhead}[1]{\textbf{#1}}
\newcommand{\code}[1]{\texttt{#1}}
\newcommand{\file}[1]{\texttt{\bfseries#1}}
\newcommand{\option}[1]{\texttt{\itshape#1}}

%----------------------------------------------------------------------------------------

\section{Background} %Body

\subsection{Automated Identification based on image traps}

Identification of objects is selecting an object among other objects based on unique features represented by one object \cite{jain2007handbook}. According to \citeauthor{jain2004introduction} \citeyear{jain2004introduction} \cite{jain2004introduction}, there are conditions attached before a feature can be suitable for use in uniquely classifying an object. Firstly, the feature need to be universal; in other words, all objects in the same class must have such a feature. Secondly, any two objects must present a unique form of the same feature, and lastly, the feature must not change over a long period of time.

 \citeauthor{duda2012pattern} \citeyear{duda2012pattern} \cite{duda2012pattern} supported the argument made about the features that are used for classification of an object and further stated that a sensing device is required to capture the signal that represents such features. Cameras are used to capture images of fingerprints so as apply computer models that extract features, and these can be used to discriminate individuals from a database containing other people's fingerprint images. \citeauthor{maltoni2009handbook} \citeyear{maltoni2009handbook} \cite{maltoni2009handbook} explain why fingerprints are suitable features in uniquely identify an individual:
 \begin{itemize}
     \item Every person in the world has fingerprints.
     \item Fingerprints are unique between at least two individuals.
     \item Fingerprints do not change over a long period of time.
 \end{itemize}
However, injuries can change fingerprint patterns. That is one of the reasons why \citeauthor{jain2004introduction} \citeyear{jain2004introduction} \cite{jain2004introduction} asserted that there is no single feature claimed to be the only most suitable feature for uniquely identifying objects. As a result, researchers have explored other features for use in the classification of individuals: iris, face, palm, and gait \cite{jain2007handbook}.     
 Automated identification has been applied in other fields like medical to detect malignant cells \cite{khan2019novel,esteva2017dermatologist}, and in computer security to detect spam emails \cite{faris2019intelligent} using text features.
 
The fields in which automated identification is applied varies greatly, however, all the previous works look at unique features that are used to discriminate objects of the same class.

\subsection{Research Problem} 

Identification of individuals using biological and behavioral features falls in the bio-metrics field \cite{jain2004introduction}. Automated individual identification has been studied extensively in human beings using face images as signals that provide discriminating features among faces of different people \cite{turk1991face}. \citeauthor{turk1991face}  \citeyear{turk1991face} \cite{turk1991face} extracted feature vectors that are relevant in distinguishing faces of different people. These features were called eigenfaces. For identification purposes, a new face is projected to the eigenfaces space, a comparison is made to see which of the eigenfaces in a database is close enough to the new face features by computing a Euclidean distance between known and new face eigenfaces. This work made assumptions that faces are usually upright, therefore the feature space was created based on upright faces. However, this assumption will only hold on face images taken from a controlled environment \cite{huang2008labeled}. This luxury can be attained with human beings, but its application is limited. Considering applying a similar approach to identifying wild animals can create problems because there is a limited chance to take a controlled face image of free-roaming Lion.     

 \citeauthor{schroff2015facenet} \citeyear{schroff2015facenet} \cite{schroff2015facenet} developed a way of identification of human beings, taking into account that face images of the same person can vary in posture, lighting, and expression. This removed the assumption that faces are usually upright. This way of extracting features could be applicable to learning features of animal face images because the posture of animal faces may vary greatly. \citeauthor{schroff2015facenet} \citeyear{schroff2015facenet} \cite{schroff2015facenet} trained a deep convolutional neural network to recognize faces from images, an approach called metric learning. An embedding is extracted from an image. Using a triplet-loss function, which takes three images; an anchor image $x_{i}^a$, a positive image $x_{i}^p$ and a negative image $x_{i}^n$. The network learns to minimize an Euclidean distance between embeddings of anchor and positive images as shown in equation \ref{tripletpos:eq}, and increases the distance between the anchor and negative image \ref{tripletneg:eq}. Several images of the same individual, portraying varying lighting conditions are required during the training phase. The positive image is any face image of the same subject, it may be a different face orientation or lighting condition. A negative image is a face image of a different subject. A hyper-parameter $\alpha$ was used to force a certain distance between the anchor image and all negative images. 
 \begin{equation}\label{tripletpos:eq}
     \parallel f(x_{i}^a)-f(x_{i}^p)\parallel^2_2
 \end{equation}
 
 \begin{equation}\label{tripletneg:eq}
     \parallel f(x_{i}^a)-f(x_{i}^n)\parallel^2_2
 \end{equation}
 
 During the training phase, the objective is to minimize the loss $L_triplet$, depicted by equation \ref{triplet:eq} adopted from \cite{schroff2015facenet}.     
 \begin{equation}\label{triplet:eq}
     L_{triplet} = \sum_{i}^N[\parallel f(x_{i}^a)-f(x_{i}^p)\parallel_{2}^2-\parallel f(x_{i}^a)-f(x_{i}^n)\parallel^2_2+\alpha] 
 \end{equation}
 
 One advantage \citeauthor{schroff2015facenet} \citeyear{schroff2015facenet} \cite{schroff2015facenet} have is the amount of training data at their disposal: over  $100$ million images from $8$ million different individuals. They reported varying accuracies on different datasets ranging between 95.18\% and 99.63\%. These results encourage one to apply the triplet-loss technique in endangered animal identification, however, the number of labeled animal images from South Africa game reserves is too low. It will be interesting to see the performance of this approach in animal data 
Automated identification of animals has been focused on species identification, with a gradual introduction of approaches to classify individuals animals \cite{norouzzadeh2018automatically}.
\citeauthor{parham2018animal} \citeyear{parham2018animal} \cite{parham2018animal} developed a pipeline for classification of species based on segmented sections of an image. The image segments called bounding boxes were compared to labeled bound boxes of six species. They capitalized on different coat patterns presented by various animal species: the coat pattern of zebras is distinct from coat patterns of giraffes. A bounding-box is labeled with a species tag if it manifests a higher degree of similarity with known coat patterns the species, compared to other species. \citeauthor{kuhl2013animal} \citeyear{kuhl2013animal} \cite{kuhl2013animal} asserted that other than just discriminating species, coat patterns can also be used as features in discriminating individuals of the same species.

Recent work has been done on identifying individual domesticated animals \cite{kumar2017real}. \citeauthor{kumar2017real} \citeyear{kumar2017real} \cite{kumar2017real} reflected on the importance of knowing where each farm cow is located, and how many unique individuals are there on a farm. The benefits of identifying an individual are efficient record-keeping about the individual, ability to track and count all known individuals, and easily recognize the missing. Computer-assisted identification of animals has reduced the manual labor from custodians of farms, and also improved the reliability of the records kept about each individual \cite{kuhl2013animal}. \citeauthor{kuhl2013animal} \citeyear{kuhl2013animal} \cite{kuhl2013animal} observed that automated identification of individual animal has far-reaching benefits in conservation research and biodiversity management. The changes of an individual due to aging or habitat can be observed to determine if descendants of the individual exhibit similar adaptations to such changes. Interventions like individualized care could be made, should there be a need, to ensure the decedents of an individual are assisted to adapt better to external changes. This cannot be achieved unless an individual can be tracked. Automated identification also presents a non-invasive way of tracking individuals compared to previous methods used in identification. \citeauthor{kuhl2013animal} \citeyear{kuhl2013animal} \cite{kuhl2013animal} further argues that collaring and implants are invasive ways of identification and may cause changes in behavior, reproduction, and ability to survive for an individual animal. These have negative effects on conservation initiatives, which may further decrease endangered species.

These benefits are relevant in assisting with the monitoring of endangered wild animals. As noted by \citeauthor{norouzzadeh2018automatically} \citeyear{norouzzadeh2018automatically} \cite{norouzzadeh2018automatically}, that conservationists will leverage on automated counting and tracking animals so as to report on biodiversity reliably, with less human effort and fewer disturbances on the habitat of wild animals. Most of the animals live in secluded spaces and human contact could potentially pause threats. There is work ongoing to keep records of species for tourist attraction and nature conservation purposes in South African reserves \cite{marnewick2008evaluating}.

\subsection{Finally}
Computer vision has evolved from using hand-engineered features for classification of images to features learned automatically by algorithms \cite{weinstein2018computer}. \citeauthor{wang2009hog} \citeyear{wang2009hog} \cite{wang2009hog} demonstrated that histogram oriented gradients (HOG) image features are good in estimating the shape of an object and local binary patterns (LBP) are good features for texture analysis. \citeauthor{wang2009hog} \citeyear{wang2009hog} \cite{wang2009hog} proposed a combination of these features in detecting human faces, leveraging on the strength of each feature: local binary patterns are robust in noisy pictures and can be used to filter irrelevant pixels from image, this will enhance HOG in detecting the shape and edges of an object in a picture since HOG suffers greatly when the image has noise. 

Researches have moved towards more adaptive methods of learning features from images. These methods involve mimicking how the human brain analyses and remembers patterns. The methods called deep neural networks have demonstrated high accuracy in image classification tasks than the histogram of gradients and local binary patterns features \cite{suleiman2017towards}. However, \citeauthor{suleiman2017towards} \citeyear{suleiman2017towards} \cite{suleiman2017towards} reflected on the fact that even though neural networks perform better, there is a trade-off; neural networks require more computational resources compared to the hand-engineered histogram of gradients. Powerful computer resources are now freely available \footnote{https://colab.research.google.com/}, these resources support experimentation that exploits the power of neural networks in image classification tasks.

\citeauthor{weinstein2018computer} \citeyear{weinstein2018computer} \cite{weinstein2018computer} shows how different configurations of neural networks have brought more improvements in the classification of wild animals based on face images. The state-of-the-art in computer vision being convolutional neural network architectures \cite{verma2018wild}.

\section{Problem Statement}
 %-----      IDEAL  -------%
Automated individual identification of endangered wildlife has been investigated using chimpanzees and Gorilla face images, and Fruit flies body images. Approaches that were employed are state-of-the-art deep neural networks which achieved plausible successes on the individual identification of these species. Automated identification enhances efforts to keep proper records about individuals in an ecosystem by reducing human errors.      
%  --------  REALITY   ---------%
However, endangered wild animals include more than just Chimpanzees. Big cats like Lions, Tigers, and Cheetahs are species threatened by extinction. Literature does not show that automated identification of individual cats has been investigated. Therefore, the process of record-keeping of other endangered species is still done manually, if at all, and is prone to human error.        
%    ----- IMPROVEMENTS------   %
In response to this problem we propose to investigate and implement automated identification of individuals in big cats population: Lions and Cheetahs using deep neural networks; with the focus of finding out if the breakthroughs made in the identification of Chimpanzees and Gorillas can be extended to other species. 

\section{Significance and Motivation}
Endangered animals in South Africa are on the decline, and efforts to curb this are being investigated \cite{marnewick2008evaluating}. Approaches to assist in this should be multi-disciplinary. While \citeauthor{marnewick2018cheetaprob} \citeyear{marnewick2018cheetaprob} \cite{marnewick2018cheetaprob} is leading data collection, counting and reporting on wildlife in major parks found in South Africa, their strategies involve collaring individuals for identification purposes, and manual image analysis. 
This work would like to investigate none-invasive means of identification, to avoid the adverse effects of collaring and other invasive identification methods robust enough to replace the manual labor involved in image analysis. 

The successes made in computer vision discipline on automated image analysis for identification purposes has brought opportunities that can be applied in individual animal identification through camera traps. It is envisioned that such approaches will improve the accuracy of reporting on wildlife demographics, and remove human error in individual animal identification. 

%%% -------------------------------My Research aims and objectives ----------------------%%%
\section{Research  aims  and Objectives}
There is a need to assist wildlife protection initiatives with reliable record-keeping of endangered animals for conservation and accurate reporting purposes. Knowing individual animals through computer-assisted identification reduces human error in individual identification, and improves the reliability of animal records. This research aims to automate individual identification of wild animals through developing and evaluating a computer model. 

The aim of this work will be achieved through the following objectives: 

\begin{itemize}

    \item To develop a deep convolutional neural network model for individual animal identification.
    \item To compare features generated by classification convolutional neural networks models with features generated by similarity learning convolutional neural networks for animal biometrics tasks.
    \item To determine which convolutional neural network architecture is best suited for individual animal classification.
    \item To develop a semi-automated animal data collection and labeling tool.
    \item To evaluate the efficiency of the deep neural network model in identifying an individual animal on the collected data.
    \item To determine the performance of the model on collected data, by measuring its mean average precision, mean reciprocal raking and accuracy. 
    \item To compare the performance of the deep neural network model with existing animal identification systems. 
    \item To analyze and write up on findings.  
\end{itemize}
%% ---------------------------%%%-------------------------------%%%-------------------------%%------------------------------
\section{Research Questions}
The main question of this research work is: 
How can deep neural networks be used to automate individual animal identification using camera trap images?

The main question has been broken down into the following sub-questions.
\begin{itemize}
    \item How do classification and similarity learning techniques compare for unique animal identification?

    \item What performance is obtained by different CNN architectures: VGG, GoogleNet, and AlexNet in unique animal identification?
\end{itemize}

\section{Delineations, Limitations and Assumptions}

This research will only focus on endangered Lion and Cheetah populations found in Kruger National Park in South Africa. The Kruger National park is a viable option because of its proximity. Time constraints have limited the work to focus on only the two species; Lion and Cheetah. More resources would be required so as to consider other endangered species.

We cannot ascertain that the outcomes presented will generalize to other endangered species found in Kruger national park and other regions.

This work relies on Human beings for labeling of individual animal images, and this labeling is assumed to be correct.

\section{Definitions}
Deep Neural networks \newline
Convolutional neural network
Siamese Neural Networks \newline
Individual \newline
Metric learning \newline

\section{Outline}
The rest of this proposal is organized as follows: \textbf{Chapter 2} discusses the research methodology adopted in the work and reasons why this methodology is relevant. Chapter 2 also gives details on tools used to collect and label data, the descriptive statistics used and metrics measured during experimentation. \textbf{Chapter 3} provides a breakdown of deliverables and time-lines for these deliverables. A Gantt chart is presented to depict milestones, the critical path, and concurrent deliverables. \textbf{Chapter 4} draws conclusions on the proposal by highlighting objectives, briefly discusses the findings and conclusions drawn from the findings.