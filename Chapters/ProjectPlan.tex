%%%%%%%%%%%%%%%%%%%%%%%%%%%%%%%%%%%%%%%%%
% Author: Terence L. van Zyl
%
% Affiliation: Wits Institute of Data Science 
%
% Template license:
% CC BY-NC-SA 3.0 (http://creativecommons.org/licenses/by-nc-sa/3.0/)
%
%%%%%%%%%%%%%%%%%%%%%%%%%%%%%%%%%%%%%%%%%
\definecolor{barblue}{RGB}{153,204,254}
\definecolor{groupblue}{RGB}{51,102,254}
\definecolor{linkred}{RGB}{165,0,33} 

\chapter{Research Plan}
\section{Introduction}
This chapter outlines the different tasks that will be undertaken to complete the research work. A set of major tasks (deliverable) is listed, and under each deliverable are sub-tasks that need to be done in order to complete each deliverable. Section \ref{deliverables} describes the deliverables, Section \ref{timePlan} depicts a time plan, showing interleaved and concurrent tasks, and Section \ref{concl} conclude the chapter.       

\section{Deliverables} \label{deliverables}

The deliverables of this research will be\footnote{All titles, journals and conferences are tentative.}:
\begin{description}
\item[Research proposal] This document needs to be completed, defended, approved and submitted. There checkpoints under the research proposal; \textit{completion of introduction chapter}, \textit{completion of methodology chapter}, \textit{defending the proposal} and \textit{submiting}. 
\item[Data collection and Experiments] Design and implementation of Web scraper and semi automated labeling tool. Develop source code for SCNN using pytorch. 
Field data collection and labeling, and conduct experiments.  
% take online turotials on how to do webcraper in python. 
% Field Data collection Collection, labeling of Lion and Cheetah corpus.
\item[Conference proceedings] ``Author Identification using Siamense Convolutional Neural Network (SCCN)'' in proceeding of IEEE IMITEC Information \& Communication Systems \/ Technologies 
%\item[Implementation of Deep neural network SCNN] Develop, train and evaluate CNN on corpus
%% p skill on how to use pytorch, running experiemnt on google colab gpus
%% architecture of cnns.  

\item[Draft dissertation] First draft write up of dissertation, introduction and conclusion. Refine methodology, findings and discussion chapters.

\end{description}

\section{Time Plan} \label{timePlan}
The gantt chart for this research proposal is available in
Table~\ref{tab:gannt_chart}.

%\begin{ganttchart}[
 % x unit=0.5mm,
 % time slot format={isodate},
 % ]{2019-07-01}{2020-06-30}
 % \gantttitlecalendar{year, month} \\
 % \ganttbar{Test}{2019-07-01}{2020-03-15}
%\end{ganttchart}

\begin{sidewaystable}[htbp]
\caption{Gantt Chart for Time Plan.}\label{tab:gannt_chart}

\newgeometry{vmargin=1cm}
%\begin{landscape}
\thispagestyle{empty}\centering
\begin{ganttchart}[vgrid={draw=none,draw=none},%
            %today=15,%
            %today offset=.5,%
            %today label=Heute,%
            %progress=today,%
            y unit title=0.7cm,
            y unit chart=0.6cm,
            bar incomplete/.append style={fill=red},%
            progress label text=  {\quad\pgfmathprintnumber[precision=0,verbatim]{#1}\%}%
            ]{1}{36}
\gantttitlecalendar*[compress calendar,time slot format=isodate]{2019-11-2}{2020-11-30}{year, month} \\
\gantttitlelist{1,...,36}{1}\\
\ganttgroup{Masters work}{1}{36} \\
 %%%%%%%%%%%%%%%%%Phase-1
\ganttgroup{Research Proposal}{1}{13} \\
\ganttbar{Scope Reading formulate topic}{1}{2} \\
\ganttbar{Literature Review}{2}{5} \ganttnewline
\ganttbar{Draft introduction chapter}{3}{7} \ganttnewline
\ganttlinkedbar{Consultation}{8}{8} \ganttnewline
\ganttlinkedbar{Make corrections introduction}{9}{9} \ganttnewline
\ganttmilestone{finalize introduction chapter}{10}\ganttnewline
\ganttbar{Draft Methodology chapter}{5}{10} \ganttnewline
\ganttlinkedbar{Consultation on methodology}{11}{11} \ganttnewline
\ganttlinkedbar{Make corrections Methodology}{11}{12} \ganttnewline
\ganttmilestone{Conclude methodology}{12}\ganttnewline
\ganttbar{Draw research plan}{10}{11} \ganttnewline
\ganttlinkedbar{prepare for Proposal defence}{12}{13}\ganttnewline
\ganttmilestone{Proposal defence}{13}\ganttnewline
\ganttlinkedbar{Make corrections}{13}{13}\ganttnewline
\ganttmilestone{Submit Proposal}{13}\ganttnewline
\ganttlink{elem6}{elem7}
\ganttlink{elem10}{elem11}
\ganttlink{elem13}{elem14}
\ganttlink{elem15}{elem16}


%%%%%%%%%%%%%%%%%Phase-2
\ganttgroup{Data \& Experimental setup}{5}{32} \\
\ganttbar{Development of data labelling tool }{5}{15} \\
\ganttbar{Development of source code SCNN}{5}{16} \ganttnewline
\ganttbar{Collect \& Label data}{6}{26} \ganttnewline 
\ganttmilestone{Finalize data set: Train and test split}{26} \ganttnewline
\ganttbar{Run training \& evaluation }{12}{28} \ganttnewline
\ganttlinkedbar{Analyze \& document results}{18}{29} \ganttnewline
\ganttmilestone{Discuss: Results with Supervisor}{30} \ganttnewline
\ganttlinkedbar{Make corrections}{31}{31} \ganttnewline
\ganttlink{elem20}{elem21}
\ganttlink{elem23}{elem24}
 %%%%%%%%%%%%%%%%%Phase-3
\ganttgroup{Conference paper}{24}{31} \\

\ganttbar{Draft Discussion of results}{24}{27} \\
\ganttlinkedbar{Publications $\&$ Workshops}{27}{30} \ganttnewline
\ganttgroup{Draft dissertation}{24}{35} \\
\ganttlinkedbar{Draft introduction \& conclusion}{30}{35} \ganttnewline
\ganttmilestone{Submit Thesis draft}{35}
\ganttlink{elem30}{elem31}
%%%%%%%%%%%%%%%%%%%%%%%%%%%%%%%%%%%%%%%%%%%%%%%%%%%%%%%%%%%%%%%
\end{ganttchart}

\end{sidewaystable}
\section{Potential Issues}\label{issues}
None
\section{Conclusion}
\label{concl}
This chapter has shown deliverables, tasks, and milestones. The time plan estimates the duration of the project work. The deliverable:  data collection and experimentation, takes most time since field trips will be undertaken to capture images of individual animals. Experimentation depends on completion of data collection and labeling. The results generated from experimentation will be documented, and this write up contributes to chapters experiments, and findings.                  
