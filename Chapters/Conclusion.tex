%%%%%%%%%%%%%%%%%%%%%%%%%%%%%%%%%%%%%%%%%
% Author: Terence L. van Zyl
%
% Affiliation: Wits Institute of Data Science 
%
% Template license:
% CC BY-NC-SA 3.0 (http://creativecommons.org/licenses/by-nc-sa/3.0/)
%
%%%%%%%%%%%%%%%%%%%%%%%%%%%%%%%%%%%%%%%%%

\chapter{Conclusion}
A literature study has shown that deep neural networks have produced a state-of-the-art performance in classification problems. Siamese convolutional neural networks, in particular, are reported to be best suited for image classification. Siamese convolutional neural networks have been applied successfully in the identification of individual human beings using face images and also in the identification of Chimpanzees. Utilizing computer-assisted individual identification has advantages in monitoring, managing and reporting on wildlife, especially endangered species.

However, there has not been work that probed how Siamese neural networks can perform in the classification of individuals from other species like Lions and Cheetahs. It is, therefore, necessary to carry out an investigation and report on how deep neural networks can assist in the classification of the big cats in-order to support reliable record-keeping about these endangered species.