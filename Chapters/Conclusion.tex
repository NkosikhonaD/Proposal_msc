%%%%%%%%%%%%%%%%%%%%%%%%%%%%%%%%%%%%%%%%%
% Author: Terence L. van Zyl
%
% Affiliation: Wits Institute of Data Science 
%
% Template license:
% CC BY-NC-SA 3.0 (http://creativecommons.org/licenses/by-nc-sa/3.0/)
%
%%%%%%%%%%%%%%%%%%%%%%%%%%%%%%%%%%%%%%%%%

\chapter{Conclusion}

\begin{theo}[In this Section]{}
The conclusion is the inverse of an introduction, so we start narrow and go wide. It provides a summary, highlights key points and pulls all the ideas together. The conclusion should include at a minimum: research objectives (a summary of your findings and the resulting conclusions); recommendations; and contributions to knowledge.

\begin{description}
\item[Research proposal] Start with the \textbf{research objectives} and answer these two questions:
\begin{enumerate}
\item After the literature review what did you find out in relation to your research objectives and questions?
\item What conclusions can YOU make?
\end{enumerate}
next offer the reader in the form of \textbf{recommendations} advice on what YOU think should happen next.

\item[Research dissertation] Start with the \textbf{research objectives} and answer these two questions:
\begin{enumerate}
\item After the literature review and the completed empirical research, what are the results in relation to your research objectives and questions?
\item What conclusions can YOU make? Do not go into analysis again!!!!
\end{enumerate}
next offer the reader in the form of \textbf{recommendations} advice on what YOU think should happen next. Finally, identify what other researchers have done and how your work builds upon theirs. This is also a good time to mention your publications and how they contributed to ``filling the gap''.

\end{description}

In summing up this section, remember that a conclusion is your last opportunity to tell the reader what you want them to remember. The chapter needs to be comprehensive and might include sub-sections.

\end{theo}
